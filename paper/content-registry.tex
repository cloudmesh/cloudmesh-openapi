\subsection{Service Registry}
\label{sec:registry}

Currently, the installation and setup of cloudmesh openapi involves the
installation of MongoDB and the configuration of mongo variables. This
is documented
\href{https://github.com/cloudmesh/cloudmesh-openapi\#installation}{here.}

There have been serveral recent cloudmesh projects involving Raspberry
Pis. Unfortunately, the minimum version of MongoDB required for openapi
is not available to the Raspberry Pi. Thus cloudmesh-openapi is not
available to Pi users with MongoDB.

In an effort to provide this software to all those that are interested
regardless of OS/machine, we have added a new default storage mechanism
that functions {\em out-of-the-box} with cloudmesh-openapi. This storage
mechanism is implemented with python's native Pickle at the heart. All
interfaces associated with MongoDB interactions have been extended to
support switching to PickleDB. Thus, this addition is backwards
compatible with previous versions of cloudmesh-openapi and requires
little changes in the existing code base to support. Since Pickle is
native to python, it is supported on any platform running python.

It is important to note that there are essentially no security
mechanisms with Pickle. We provide this option for users to test their
APIs on different machines with little to no setup, but we do not
recommend its usage in a production server.

See Appendix \ref{a.6-switching-between-pickledb-and-mongodb} 
to see how to switch between DB protocols.

\subsection{A.6 Switching between PickleDB and
MongoDB}\label{a.6-switching-between-pickledb-and-mongodb}

The default ``out-of-the-box'' storage mechanism of cloudmesh-openapi is
Pickle. This requires no setup of the DB on the user's end.

To switch to MongoDB, the user must first change their config option as
follows:

\begin{verbatim}
cms openapi register protocol mongo
\end{verbatim}

Note that by switching to mongo, certain mongo variables need to be
filled out. Mongo may need to be installed as well. Refer to
\href{https://github.com/cloudmesh/cloudmesh-openapi/\#installation}{this}
documentation to see how this process can be done.

One may switch back to pickle with the same command:

\begin{verbatim}
cms openapi register protocol pickle
\end{verbatim}

\subsection{APPENDIX B. - Code
Location}\label{appendix-b.---code-location}

This is temporary and will in final be moved elsewhere. Its conveniently
for now placed on top so we can easier locate it at \cite{cloudmesh-openapi}.

