Not only will we be updating this code to leverage new
Python 3 features, but we will also reorganize the code into several
repositories. To do so, we will build repository bundles that
interfaces to a cloud provider specifically. This allows us to create
sophisticated plugins that target a particular cloud provider or
service. As a consequence users can choose which providers or services
they like to work with and only install the once needed. Through
package dependencies, all needed packages for a particular provider or
service can be installed with a single install command.  This will
drastically reduce the time needed for management and deployment by
the users.

The other aspect of this effort is the use of high-level services that
are already existing in the clouds targeting specific analytics tasks
such as natural language processing and image classification. Here
services may already exist as ready to be used components in the cloud
and can be integrated into the workflow of the users as part of the
generalized analytics services. However, as each vender tries to
promote their own technologies, we will explore the creation of a
high-level service-independent analytics service multiplexer that is
capable of delegating the analysis task to a specific cloud
service. If multiple such services on different providers exist, it
can be coupled with a scheduler framework providing a selection a
capability which of them is to be used.

Lastly, it is obvious that we can showcase these capabilities within
Jupyter notebooks, as discussed in Section \ref{sec:jupyter}.
